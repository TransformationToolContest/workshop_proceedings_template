%!TEX root=./proceedings.tex

%% Event and proceedings details

\proceedings{9th International Workshop}{Transformation Tool Contest}{TTC 2016}
\held{8th July, 2016}{Vienna, Austria}

\editor{Antonio Garcia-Dominguez}
\editor{Filip Krikava}
\editor{Louis Rose}

%% Colours for the front and back pages

\definecolor{titlepagecolor}{cmyk}{1,.60,0,.40}
\definecolor{namecolor}{cmyk}{1,.50,0,.10} 

%% Front matter

\license{Copyright \copyright\ 2016 for the individual papers by the papers' authors. Copying permitted only for private and academic purposes. This volume is published and copyrighted by its editors.}

\preface{
The aim of the Transformation Tool Contest (TTC) series is to compare the expressiveness, the usability, and the performance of transformation tools along a number of selected case studies.
A deeper understanding of the relative merits of different tool features will help to further improve transformation tools and to indicate open problems.

This contest was the ninth of its kind.
For the fourth time, the contest was part of the Software Technologies: Applications and Foundations (STAF) federation of conferences.
Teams from the major international players in transformation tool development have participated in an online setting as well as in a face-to-face workshop.

In order to facilitate the comparison of transformation tools, our programme committee
selected the following challenging case via single blind reviews: Class Responsibility Assignment Case, for which eventually eight solutions were accepted.

These proceedings comprise descriptions of the three cases and descriptions of all of the
solutions to these cases.
In addition to the solution descriptions contained in these proceedings, the implementation of each solution (tool, project files, documentation) is made available for review and demonstration via the SHARE platform (http://share20.eu).

TTC 2016 involved open (i.e., non anonymous) peer reviews in a first round.
The purpose of this round of reviewing was that the participants gained as much insight into
the competitors’ solutions as possible and also to identify potential problems.
At the workshop, the solutions were presented.
The expert audience judged the solutions along a number of case-specific categories, and prizes were awarded to the highest scoring solutions in each category.
A summary of these results for each case are included in these proceedings.
Finally, the solutions appearing in these proceedings were selected by our programme committee via single blind reviews. The full results of the contest are published on our website\footnote{\url{http://www.transformation-tool-contest.eu/}}.

Besides the presentations of the submitted solutions, the workshop also comprised a live contest.
That contest involved creating a solution for dataflow-based model transformations.
The live contest was announced to all STAF attendees and participants were given four days to design, implement and test their solutions.
The contest organisers thank all authors for submitting cases and solutions, the contest
participants, the STAF local organisation team, the STAF general chair Gerti Kappel, and the program committee for their support.
}

%% Organisation page

\organisation{PreProTeX 2014 would have been organised by the Department of Computer Science, at the University of York (had it not been entirely imaginary).}

\pcmember{David Reviewer}{IT University of Copenhagen, Denmark}
\pcmember{Michael Harsh}{Politecnico di Milano, Italy}

\additionalreviewer{James Overworked}{Vienna University of Technology, Austria}

%% Body of the proceedings

% \paper takes four arguments:
%   id -- an integer such that "./papers/paper" + id + ".pdf" gives the path to this paper
%   title -- the title of the paper
%   authors -- the complete set of authors of the paper
%   page -- the starting page of the paper in the proceedings
\paper{9}{A Truly Excellent Paper}{Louis M. Rose}{1}
\paper{4}{A Less Good Paper}{John Writer}{6}

%% Alternatively, the body can be divided into different parts:
% \proceedingspart{Research Track}
% \paper{9}{A Truly Excellent Paper}{Louis M. Rose}{1}
%
% \proceedingspart{Tools Track}
% \paper{4}{A Less Good Paper}{John Writer}{6}

