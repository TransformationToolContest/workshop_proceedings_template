%!TEX root=./proceedings.tex

%% Event and proceedings details

\proceedings{11th International Workshop}{Transformation Tool Contest}{TTC 2018}
\held{29th June, 2018}{Toulouse, France}

\editor{Antonio Garcia-Dominguez}
\editor{Georg Hinkel}
\editor{Filip K\v{r}ikava}

%% Colours for the front and back pages

\definecolor{titlepagecolor}{cmyk}{1,.60,0,.40}
\definecolor{namecolor}{cmyk}{1,.50,0,.10}

%% Front matter

\license{Copyright \copyright\ 2018 for the individual papers by the papers'
authors. Copying permitted only for private and academic purposes. This volume
is published and copyrighted by its editors.}

\preface{The aim of the Transformation Tool Contest (TTC) series is to compare
the expressiveness, the usability, and the performance of transformation tools
along a number of selected case studies. A deeper understanding of the relative
merits of different tool features will help to further improve transformation
tools and to indicate open problems.

This contest was the tenth of its kind. For the fifth time, the contest was part
of the Software Technologies: Applications and Foundations (STAF) federation of
conferences. Teams from the major international players in transformation tool
development have participated in an online setting as well as in a face-to-face
workshop.

In order to facilitate the comparison of transformation tools, our programme
committee selected three challenging cases via single blind reviews for which
there were together ten solutions.

These proceedings comprise descriptions of the case study and all of the
accepted solutions. In addition to the solution descriptions contained in these
proceedings, the implementation of each solution (tool, project files,
documentation) is made available in public version control repositories. Some
solution are also available via the SHARE platform (http://share20.eu).

TTC 2018 involved open (i.e., non anonymous) peer reviews in a first round. The
purpose of this round of reviewing was that the participants gained as much
insight into the competitors’ solutions as possible and also to identify
potential problems. At the workshop, the solutions were presented. The expert
audience judged the solutions along a number of case-specific categories, and
prizes were awarded to the highest scoring solutions in each category. Finally,
the solutions appearing in these proceedings were selected by our programme
committee via single blind reviews. The full results of the contest are
published on our
website\footnote{\url{http://www.transformation-tool-contest.eu/}}.

Besides the presentations of the submitted solutions, the workshop also
comprised a live contest. That contest involved creating a solution for
transformation reuse in the presence of multiple inheritance and redefinitions.
The live contest was announced to all STAF attendees and participants were given
four days to design, implement and test their solutions. The contest organisers
thank all authors for submitting cases and solutions, the contest participants,
the STAF local organisation team, the STAF general chair Gabriele Taentzer, and
the program committee for their support. }

%% Organisation page

\organisation{The Transformation Tool Contest has been organized by IRIT in Toulose, France.}

\pcmember{Konstantinos Barmpis}{University of York, United Kingdom}
\pcmember{Juan Boubeta-Puig}{University of Cádiz, Spain}
\pcmember{Erwan Bousse}{Vienna University of Technology, Austria}
\pcmember{Philippe Collet}{Université Nice Sophia-Antipolis, France}
\pcmember{Gwendal Daniel}{AtlanMod - Inria, France}
\pcmember{Coen De Roover}{Vrije Universiteit Brussel, Belgium}
\pcmember{Antonio Garcia-Dominguez}{Aston University, United Kingdom}
\pcmember{Georg Hinkel}{FZI Research Center of Information Technology, Germany}
\pcmember{Tassilo Horn}{SHD, Germany}
\pcmember{Akos Horvath}{Budapest University of Technology and Economics, Hungary}
\pcmember{Christian Krause}{SAP Innovation Center, Germany}
\pcmember{Filip K\v{r}ikava}{Czech Technical University, Czech Republic}
\pcmember{Arend Rensink}{University of Twente, The Netherlands}
\pcmember{Sonja Schimmler}{Bundeswehr University Munich, Germany}
\pcmember{Massimo Tisi}{Ecole des Mines de Nantes, France}
\pcmember{Gergely Varro}{Technische Universitat Darmstadt, Germany}
\pcmember{Ran Wei}{University of York, United Kingdom}
\pcmember{Bernhard Westfechtel}{University of Bayreuth, Germany}

%% Body of the proceedings

% \paper takes four arguments:
%   id -- an integer such that "./papers/paper" + id + ".pdf" gives the path to this paper
%   title -- the title of the paper
%   authors -- the complete set of authors of the paper
%   page -- the starting page of the paper in the proceedings
\proceedingspart{Quality-based Software Selection and Hardware-Mapping Case}
\paper{1}{Quality-based Software-Selection and Hardware-Mapping as Model Transformation Problem}{Sebastian Götz, Johannes Mey, René Schöne and Uwe Aßmann}{3}
\paper{2}{The EMFeR Solution to the TTC 2018 Software Mapping Case}{Christoph Eickhoff, Simon-Lennert Raesch and Albert Zündorf}{13}
\paper{3}{Solving the Quality-based Software-Selection and Hardware-Mapping Problem with ACO}{Samaneh Hoseindoost, Meysam Karimi, Shekoufeh Kolahdouz-Rahimi and Bahman Zamani}{19}
\paper{4}{A JastAdd- and ILP-based Solution to the Software-Selection and Hardware-Mapping-Problem at the TTC 2018}{Sebastian Götz, Johannes Mey, René Schöne and Uwe Aßmann}{31}

\proceedingspart{Social Media Live Case}
\paper{5}{The TTC 2018 Social Media Case}{Georg Hinkel}{39}
\paper{6}{An NMF solution to the TTC 2018 Social Media Case}{Georg Hinkel}{45}
\paper{7}{Hawk solutions to the TTC 2018 Social Media Case}{Antonio Garcia-Dominguez}{51}
\paper{8}{A JastAdd-based Solution to the TTC 2018 Social Media Case}{René Schöne and Johannes Mey}{57}
\paper{9}{YAMTL Solution to the TTC 2018 Social Media Case}{Artur Boronat}{65}
\paper{10}{The TTC 2018 Social Media Case, by ATL and AOF}{Valentin Besnard, Frédéric Jouault, Théo Le Calvar and Massimo Tisi}{79}

%% Alternatively, the body can be divided into different parts:
% \proceedingspart{Research Track}
% \paper{9}{A Truly Excellent Paper}{Louis M. Rose}{1}
%
% \proceedingspart{Tools Track}
% \paper{4}{A Less Good Paper}{John Writer}{6}
