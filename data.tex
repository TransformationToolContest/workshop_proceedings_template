%!TEX root=./proceedings.tex

%% Event and proceedings details

\proceedings{100th International Workshop}{Preparing Proceedings in LaTeX}{PreProTeX 2014}
\held{30th September, 2014}{York, United Kingdom}

\editor{Sally Editor}
\editor{John Writer}

%% Colours for the front and back pages

\definecolor{titlepagecolor}{cmyk}{1,.60,0,.40}
\definecolor{namecolor}{cmyk}{1,.50,0,.10} 

%% Front matter

\license{Copyright \copyright\ 2014 for the individual papers by the papers' authors. Copying permitted only for private and academic purposes. This volume is published and copyrighted by its editors.}

\preface{Your preface goes here. These proceedings serve as a timely reminder of the wonderful time had by all in attending this entirely imaginary workshop.

The preface can be more than one paragraph.}

%% Organisation page

\organisation{PreProTeX 2014 would have been organised by the Department of Computer Science, at the University of York (had it not been entirely imaginary).}

\pcmember{David Reviewer}{IT University of Copenhagen, Denmark}
\pcmember{Michael Harsh}{Politecnico di Milano, Italy}

\additionalreviewer{James Overworked}{Vienna University of Technology, Austria}

%% Body of the proceedings

% \paper takes four arguments:
%   id -- an integer such that "./papers/paper" + id + ".pdf" gives the path to this paper
%   title -- the title of the paper
%   authors -- the complete set of authors of the paper
%   page -- the starting page of the paper in the proceedings
\paper{9}{A Truly Excellent Paper}{Louis M. Rose}{1}
\paper{4}{A Less Good Paper}{John Writer}{6}

%% Alternatively, the body can be divided into different parts:
% \proceedingspart{Research Track}
% \paper{9}{A Truly Excellent Paper}{Louis M. Rose}{1}
%
% \proceedingspart{Tools Track}
% \paper{4}{A Less Good Paper}{John Writer}{6}

