%!TEX root=./proceedings.tex

%% Event and proceedings details

\proceedings{10th International Workshop}{Transformation Tool Contest}{TTC 2017}
\held{21th July, 2017}{Marburg, Germany}

\editor{Antonio Garcia-Dominguez}
\editor{Georg Hinkel}
\editor{Filip K\v{r}ikava}

%% Colours for the front and back pages

\definecolor{titlepagecolor}{cmyk}{1,.60,0,.40}
\definecolor{namecolor}{cmyk}{1,.50,0,.10}

%% Front matter

\license{Copyright \copyright\ 2017 for the individual papers by the papers' authors. Copying permitted only for private and academic purposes. This volume is published and copyrighted by its editors.}

\preface{
The aim of the Transformation Tool Contest (TTC) series is to compare the expressiveness, the usability, and the performance of transformation tools along a number of selected case studies.
A deeper understanding of the relative merits of different tool features will help to further improve transformation tools and to indicate open problems.

This contest was the tenth of its kind.
For the fifth time, the contest was part of the Software Technologies: Applications and Foundations (STAF) federation of conferences.
Teams from the major international players in transformation tool development have participated in an online setting as well as in a face-to-face workshop.

In order to facilitate the comparison of transformation tools, our programme committee selected three challenging cases via single blind reviews for which there were together ten solutions. 

These proceedings comprise descriptions of the case study and all of the accepted solutions.
In addition to the solution descriptions contained in these proceedings, the implementation of each solution (tool, project files, documentation) is made available in public version control repositories.
Some solution are also available via the SHARE platform (http://share20.eu).

TTC 2017 involved open (i.e., non anonymous) peer reviews in a first round.
The purpose of this round of reviewing was that the participants gained as much insight into the competitors’ solutions as possible and also to identify potential problems.
At the workshop, the solutions were presented.
The expert audience judged the solutions along a number of case-specific categories, and prizes were awarded to the highest scoring solutions in each category.
Finally, the solutions appearing in these proceedings were selected by our programme committee via single blind reviews.
The full results of the contest are published on our website\footnote{\url{http://www.transformation-tool-contest.eu/}}.

Besides the presentations of the submitted solutions, the workshop also comprised a live contest.
That contest involved creating a solution for transformation reuse in the presence of multiple inheritance and redefinitions.
The live contest was announced to all STAF attendees and participants were given four days to design, implement and test their solutions.
The contest organisers thank all authors for submitting cases and solutions, the contest participants, the STAF local organisation team, the STAF general chair Gabriele Taentzer, and the program committee for their support.
}

%% Organisation page

\organisation{The Transformation Tool Contest has been organized by Philipps-Universität Marburg, Germany.}

\pcmember{Olivier Barais}{University of Rennes 1, France}
\pcmember{Philippe Collet}{Université Nice Sophia-Antipolis, France}
\pcmember{Coen De Roover}{Vrije Universiteit Brussel, Belgium}
\pcmember{Antonio Garcia-Dominguez}{Aston University, United Kingdom}
\pcmember{Jeff Gray}{University of Alabama, United States}
\pcmember{Tassilo Horn}{SHD, Germany}
\pcmember{Akos Horvath}{Budapest University of Technology and Economics, Hungary}
\pcmember{Christian Krause}{SAP Innovation Center, Germany}
\pcmember{Filip K\v{r}ikava}{Czech Technical University, Czech Republic}
\pcmember{Sonja Schimmler}{Bundeswehr University Munich, Germany}
\pcmember{Arend Rensink}{University of Twente, The Netherlands}
\pcmember{Louis Rose}{University of York, United Kingdom}
\pcmember{Bernhard Schatz}{Technische Universitat Munchen, Germany}
\pcmember{Massimo Tisi}{Ecole des Mines de Nantes, France}
\pcmember{Tijs van der Storm}{Centrum Wiskunde \& Informatica, The Netherlands}
\pcmember{Pieter Van Gorp}{Eindhoven University of Technology, The Netherlands}
\pcmember{Gergely Varro}{Technische Universitat Darmstadt, Germany}
\pcmember{Bernhard Westfechtel}{University of Bayreuth, Germany}

%% Body of the proceedings

% \paper takes four arguments:
%   id -- an integer such that "./papers/paper" + id + ".pdf" gives the path to this paper
%   title -- the title of the paper
%   authors -- the complete set of authors of the paper
%   page -- the starting page of the paper in the proceedings
\proceedingspart{Smart Grid Case}
\paper{1}{The TTC 2017 Outage System Case for Incremental Model Views}{Georg Hinkel}{3}
\paper{5}{An NMF solution to the Smart Grid Case at the TTC 2017}{Georg Hinkel}{13}
\paper{17}{Detecting and Preventing Power Outages in a Smart Grid using eMoflon}{Sven Peldszus, Jens Bürger and Daniel Strüber}{19}

\proceedingspart{Families to Persons Case}
\paper{2}{The Families to Persons Case}{Anthony Anjorin, Thomas Buchmann and Bernhard Westfechtel}{27}
\paper{6}{An NMF solution to the Families to Persons case at the TTC 2017}{Georg Hinkel}{35}
\paper{9}{The SDMLib Solution to the TTC 2017 Families 2 Persons Case}{Albert Zuendorf and Alexander Weidt}{41}
\paper{13}{Solving the TTC Families to Persons Case with FunnyQT}{Tassilo Horn}{47}
\paper{14}{Solving the Families to Persons Case using EVL+Strace}{Leila Samimi-Dehkordi, Bahman Zamani and Shekoufeh Kolahdouz-Rahimi}{53}

\proceedingspart{State Elimination Case}
\paper{4}{State Elimination as Model Transformation Problem}{Sinem Getir, Duc Anh Vu, Francois Peverali, Daniel Strüber and Timo Kehrer}{65}
\paper{7}{An NMF solution to the State Elimination case at the TTC 2017}{Georg Hinkel}{75}
\paper{10}{Transformation of Finite State Automata to Regular Expressions using Henshin}{Daniel Strüber}{81}
\paper{11}{The SDMLib solution to the TTC 2017 State Elimination Case}{Alexander Weidt and Albert Zuendorf}{87}
\paper{12}{Solving the State Elimination Case Study using Epsilon}{Mohammadreza Sharbaf, Shekoufeh Kolahdouz-Rahimi and Bahman Zamani}{91}


%% Alternatively, the body can be divided into different parts:
% \proceedingspart{Research Track}
% \paper{9}{A Truly Excellent Paper}{Louis M. Rose}{1}
%
% \proceedingspart{Tools Track}
% \paper{4}{A Less Good Paper}{John Writer}{6}
